% !Mode:: "TeX:UTF-8" 

\chapter{夫君子之行静以修身俭以养德非淡泊无以明志非宁静无以致远夫学须静也才须学也非学无以广才非志无以成学淫漫则不能励精险躁则不能冶性年与时驰意与日去遂成枯落多不接世悲守穷庐将复何及}[Introduction]

该文件中有些字体默认的设置会导致字体出错,例如无法找到[SimKai]之类。解决方法也很简单,要么在这个文件里面设置好字体;
该文件中有些字体默认的设置会导致字体出错,例如无法找到[SimKai]之类。解决方法也很简单,要么在这个文件里面设置好字体;
要么在setup/package.tex中自定义好字体。
\section{XeLaTeX编译方法的配置输入谷关紧要的汉子占地方其实全是废话遮掩}[Introduction to the \XeLaTeX way of compiling]
该文件中有些字体默认的设置会导致字体出错,例如无法找到[SimKai]之类。解决方法也很简单,要么在这个文件里面设置好字体;
\sindex[english]{shuhalo}
\sindex[china]{an!安装}
\sindex[english]{else}
该文件中有些字体默认的设置会导致字体出错,例如无法找到[SimKai]之类。解决方法也很简单,要么在这个文件里面设置好字体;
该文件中有些字体默认的设置会导致字体出错,例如无法找到[SimKai]之类。解决方法也很简单,要么在这个文件里面设置好字体;
要么在setup/package.tex中自定义好字体。
\sindex[china]{an!安培计}
要么在setup/package.tex中自定义好字体。
要么在setup/package.tex中自定义好字体。
\subsection{模板的使用方法介绍}[Introduction to the application method of the template]
该文件中有些字体默认的设置会导致字体出错,例如无法找到[SimKai]之类。解决方法也很简单,要么在这个文件里面设置好字体;
要么在setup/package.tex中自定义好字体。
要么在setup/package.tex中自定义好字体。
要么在setup/package.\cite{kanamori1998shaking}tex中自定义好字体。
要么在setup/package.tex中自定义好字体。
要么在setup/package\cite{chen2016real}.tex中自定义好字体。
\subsubsection{模板的使用方法介绍}[Introduction to the application method of the template]
模版无法用\XeLaTeX 编译的原因主要是~\href{http://bay.uchicago.edu/tex-archive/macros/xetex/latex/xecjk/xeCJK.pdf}{xeCJK}包不兼容CJK包。
例如:xeCJK原来重写了CJK包大部分命令,如\textbackslash xeCJKcaption 。
例如:xeCJK原来重写了CJK包大部分命令,如\textbackslash xeCJKcaption 。
