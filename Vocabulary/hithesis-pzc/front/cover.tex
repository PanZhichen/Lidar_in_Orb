% !Mode:: "TeX:UTF-8" 

\hitsetup{
  %******************************
  % 注意:
  %   1. 配置里面不要出现空行
  %   2. 不需要的配置信息可以删除
  %******************************
  %
  %=====
  % 秘级
  %=====
  statesecrets={公开},
  natclassifiedindex={TM301.2},
  internatclassifiedindex={62-5},
  %
  %=========
  % 中文信息
  %=========
  ctitleone={室外场景下的},
  ctitletwo={三维视觉-激光SLAM系统和基于ClusterMap的重定位方法},
  ctitle={室外场景下的三维视觉-激光SLAM系统和基于\textit{ClusterMap}的重定位方法},
  cxueke={工学},
  csubject={控制工程},
  caffil={哈尔滨工业大学(深圳)},
  cauthor={潘志琛},
  csupervisor={陈浩耀副教授},
  %cassosupervisor={某某某教授}, % 副指导老师
  %ccosupervisor={某某某教授}, % 联合指导老师
  % 日期自动使用当前时间,若需指定按如下方式修改:
  cdate={2018年11月},
  %cstudentid={16S053099},
  %
  %
  %=========
  % 英文信息
  %=========
  etitle={3D Visual-Lidar SLAM and \textit{ClusterMap} Based Re-Localization Approach in Outdoor Scenes},
  % 这块比较复杂,需要分情况讨论:
  % 1. 学术型硕士
  %    edegree:必须为Master of Arts或Master of Science(注意大小写)
  %             “哲学、文学、历史学、法学、教育学、艺术学门类,公共管理学科
  %              填写Master of Arts,其它填写Master of Science”
  %    emajor:“获得一级学科授权的学科填写一级学科名称,其它填写二级学科名称”
  % 2. 专业型硕士
  %    edegree:“填写专业学位英文名称全称”
  %    emajor:“工程硕士填写工程领域,其它专业学位不填写此项”
  % 3. 学术型博士
  %    edegree:Doctor of Philosophy(注意大小写)
  %    emajor:“获得一级学科授权的学科填写一级学科名称,其它填写二级学科名称”
  % 4. 专业型博士
  %    edegree:“填写专业学位英文名称全称”
  %    emajor:不填写此项
  exueke={Engineering},
  esubject={Control Engineering},
  eaffil={Harbin Institute of Technology(Shenzhen)},
  eauthor={Pan Zhichen},
  esupervisor={Professor Chen Haoyao},
  %eassosupervisor={Chen Wenguang},
  % 日期自动生成,若需指定按如下方式修改:
  edate={November, 2018},
  %
  % 关键词用“英文逗号”分割
  ckeywords={三维视觉-激光SLAM, 室外, 聚类地图, 重定位},
  ekeywords={3D visual-lidar SLAM, outdoor, \textit{ClusterMap}, re-localization},
}

\begin{cabstract}
  SLAM(Simultaneous Localization and Mapping)是空间定位技术的前沿方向,按照不同的传感器,可以大致分为视觉SLAM和激光SLAM两类,但无论哪种传感器,都有其局限性。例如,单目视觉SLAM存在固有尺度偏差,会造成定位误差的累积,并且容易受到光照变化的不利影响。只利用激光雷达作为传感器的SLAM系统会受制于激光检测距离的有限性,且从环境中提取稳定特征信息较为困难,尤其是在室外、非结构化的场景中。因此本文提出了一种框架完整的三维视觉-激光SLAM系统,称为VL-SLAM,该系统融合单目相机和三维激光传感器数据,能够在室外环境下,获得与真实场景尺度一致的定位与建图结果。此外,本文利用从三维激光点云数据中提取的目标物体的点云簇,生成聚类地图,称为\textit{ClusterMap},并在该地图中,根据相邻点云簇之间的相互位置关系建立每个点云簇的位置描述子,然后通过匹配位置描述子,可以实现重新定位。这种重定位的方法对三维点云的稠密度要求不高,且不需要对点云实现完美的分割,而且可以忽略一些诸如环境光照、物体形貌、观测方向等条件变化对重定位算法可靠度的影响,这些影响在依赖于视觉特征或者物体外观特征实现重定位的方法中是很难消除的。

本文同时在KITTI数据集和自制数据集上对算法性能做了验证,结果表明,文中提出的三维视觉-激光SLAM能够在室外环境下获得低漂移的定位和建图结果,且基于聚类地图的重定位方法能够可靠地利用已有地图实现重新定位,即使该地图是数月之前生成的。
\end{cabstract}

\begin{eabstract}
SLAM (Simultaneous Localization and Mapping) is the frontier of spatial locating technology. According to different sensors, it can be roughly divided into two types: visual SLAM and laser SLAM, but any sensor has its limitations. For example, SLAM based solely on monocular camera suffers from scale inconsistencies and illumination changes while lidar-only SLAM method suffers from limited detection distance and difficulties to extract robust features, especially in outdoor, unstructured environment, in this paper, we propose a complete 3D visual-lidar SLAM system, namely VL-SLAM, which combines a monocular camera with a lidar and can obtain scale-invariant localization and mapping results in outdoor scenes. In addition, we extract point clusters of interesting objects in lidar perceptions to generate \textit{ClusterMap}, based on which we propose a re-localization method using location descriptors created according to the mutual locational relationships between adjacent clusters. This method does not depend on the high density and perfect clustering of point cloud and can eliminate effects of changes on such as illumination, appearance, and viewpoint which are difficult to overcome when using visual features or object characteristics to achieve re-localization.

We verify the performance of proposed algorithms on both KITTI dataset and customized dataset where the results show that the method demonstrated in this paper can acquire low-drift localization and mapping results, and able to reliably complete re-localization using pre-built \textit{ClusterMap}, even though it is built a few months ago.
\end{eabstract}
